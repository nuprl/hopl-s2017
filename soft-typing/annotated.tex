\documentclass{article}
\usepackage[colorlinks=true]{hyperref}
\usepackage{amsmath}
\usepackage{amssymb}
\usepackage{amsthm}
\usepackage{dashrule}
\usepackage{tikz}
\usetikzlibrary{shapes}
\usepackage{textgreek}
\usepackage{multicol}
\usepackage{listings}

\renewcommand{\maketitle}[1]{{%
\flushleft\textsf{#1 \hfill \today} \\
\textsf{\hfill Ben Greenman} \\
\vspace{1mm}
\hrule
\vspace{0.4cm}}}


\newcommand{\hdash}{{\vspace{1ex}}\hdashrule{\linewidth}{0.2ex}{1em}{\vspace{1ex}}}
\newcommand{\naturals}{\mathbb{N}}

\newcommand{\localpdf}[2]{#2} % links to local pdfs not working, oh well
\newcommand{\figref}[1]{Figure~\ref{#1}}
\newcommand{\secref}[1]{Section~\ref{#1}}

\newtheorem*{theorem}{Theorem}

\newcommand{\qa}[2]{\vspace{2ex}\noindent \textbf{Q.} #1 \newline \noindent \textbf{A.} #2}

\renewcommand{\maketitle}[2]{\begin{center}{\large \textsf{#1} } \\[1ex] \textsf{#2, \href{http://www.ccs.neu.edu/home/matthias/7480-s17/index.html}{HOPL 2017}} \end{center}}
\begin{document}
\maketitle{Soft Typing}{Ben Greenman}

\begin{abstract}
  A \emph{soft type checker} for a language $\mathsf{L}$ analyzes $\mathsf{L}$
  programs and inserts casts to ensure runtime type safety.
  A soft type checker does not reject programs as ``ill-typed'', nor does it
  require annotations that would not appear in typical $\mathsf{L}$ programs.
\end{abstract}

\subsubsection*{Soft typing: An approach to type checking for dynamically typed languages}
\begin{verbatim}
@phdthesis{f-thesis-1991,
  author = {Mike Fagan},
  title = {Soft typing: An approach to type checking for
           dynamically typed languages},
  school = {Rice University},
  year = {1991}
}
\end{verbatim}

Defines a soft type system as a source-to-source translator that:
\begin{itemize}
\item does not reject any syntactically-correct programs,
\item assures the run-time safety of its output via inserted assertions,
\item infers and checks types ``unobtrusively''.
\end{itemize}

To clarify ``unobtrusive'', Fagan defines \emph{minimal text} and \emph{minimal
failure} principles:
\begin{description}
\item[Minimal Text Principle]
  \ldots the type checking system should function in the absence of
  programmer-supplied type declarations.
\item[Minimal Failure Principle]
  The checker must pass ``a large fraction'' of dynamic programs that cannot
  be verified.
\end{description}

In this spirit, Fagan defines a type algebra with untagged unions and recursive
types, along with a set of type inference rules that derive a valid typing
for any program written in a small functional language.

% \emph{Is it a soft type system?} yes.

\emph{Significance:} this dissertation introduced soft typing.
 The idea that a type checker could guarantee the \emph{safety} of programs
 it could not prove type-correct was novel.


\newpage
\subsubsection*{Typechecking records and variants in a natural extension of ML}
\begin{verbatim}
@inproceedings{r-popl-1989,
  author = {Didier R\'{e}my},
  title = {Typechecking records and variants in a natural
           extension of ML},
  booktitle = {POPL},
  pages = {77 -- 88},
  year = {1989}
}
\end{verbatim}

R\'{e}my introduces a technique for encoding structural subtyping on record
types through polymorphism.
The key idea is to represent record types as a sequence of labeled
\emph{and flagged} types.
Flags indicate whether a given label is present or absent.
By being polymorphic over the flags it accepts, a function can, e.g., accept
records with more present fields that it explicitly accesses.

% \emph{Is it a soft type system?} no.

\emph{Significance (to soft typing):} introduced the encoding that Fagan
and Wright later used.


\newpage
\subsubsection*{Practical Soft Typing}
\begin{verbatim}
@phdthesis{w-thesis-1994,
  author = {Andrew K. Wright},
  title = {Practical Soft Typing},
  school = {Rice University},
  year = {1994}
}
\end{verbatim}

Adapts Fagan's soft type system (for a small functional language) to R4RS
Scheme, both in theory and with an implementation.
On the theoretical side, Wright's type system represents types more compactly
and accomodates features including user-defined types, mutable state (with
strong updates), {\tt call/cc}, and recursive definitions.
On the practical side, Wright demonstrates that the implementation can
infer usefully-precise types for small programs and is often able to improve
programs' performance by removing runtime checks.

% \emph{Is it a soft type system?} yes.

\emph{Significance:} Wright's dissertation describes the first (and possibly,
only) implementation of soft typing for a widely-used language.


\newpage
\subsubsection*{Soft Typing with Conditional Types}
\begin{verbatim}
@inproceedings{awl-popl-1994,
  author = {Alexander Aiken
            and Edward L. Wimmers
            and T. K. Lakshman},
  title = {Soft Typing with Conditional Types},
  booktitle = {POPL},
  pages = {163 -- 173},
  year = {1994}
}
\end{verbatim}

Describes a soft type system where types are sets and type inference builds a
collection of set inclusion constraints.
On one hand, this approach infers precise types for unannotated programs
and the type system can model control flow with set-union and set-intersection
operations.
On the other hand, a program can generate a very large collection of
constraints and the authors' constraint solving algorithm takes exponential
time in the worst case.
The paper reports low time and memory overhead (at most 30 seconds and 5 MB)
on programs with ``hundreds of lines'', but it is not clear whether the
technique will work at scale.

%\emph{Is it a soft type system?} yes, the system infers precise types for
% \mathsf{FL} programs without help from user-supplied annotations.
% Furthermore, the system inserts runtime checks for programs it cannot prove
% are statically well-typed.

\emph{Significance:} this work explores soft typing with explicit
 constraints, as opposed to unification over equations with slack variables.
 The \emph{conditional types} later influenced Sam Tobin-Hochstadt's work on
 occurrence typing.


\newpage
\subsubsection*{Quasi-Static Typing}
\begin{verbatim}
@inproceedings{t-popl-1990,
  author = {Satish Thatte},
  title = {Quasi-Static Typing},
  booktitle = {POPL},
  pages = {367 -- 381},
  year = {1990}
}
\end{verbatim}

Thatte's quasi-static type system is a variant of the simply-typed lambda
calculus with implicit, unrestricted dynamic typing.
In particular, the type system includes a ``Dynamic'' type representing a lack
of type information.
Programmers may use ``Dynamic'' as a type annotation; the quasi-static type
checker implicitly adds casts to and from the dynamic type.
Additionally, Thatte defines a ``plausibility checker'' that removes redundant
casts and replaces impossible casts (e.g., from $\mathsf{Int}$ to
$\mathsf{(Int \rightarrow Int)}$) with an error term.

% \emph{Is it a soft type system?} 

\emph{Significance:} Thatte's idea that ``possible dynamic typing as a property
should be inherited'' was novel, and appears later in Siek's work on gradual
typing.


\newpage
\subsubsection*{Global Tagging Optimization by Type Inference}
\begin{verbatim}
@inproceedings{h-lfp-1992,
  author = {Fritz Henglein},
  title = {Global Tagging Optimization by Type Inference},
  booktitle = {LFP},
  pages = {205 -- 215},
  year = {1992}
}
\end{verbatim}

Dynamically typed programs ensure type safety at runtime via implicit,
first-order casts.
For some of these casts, one can show that they never fail because they never
receive ill-typed data.
Henglein does so with a type inference algorithm.
The algorithm makes these casts (implied by data destructors) and first-order
datatypes (implied by data constructors) explicit.
Together, these explicit casts describe a system of constraints that Henglein
solves in almost-linear time to identify unnecessary casts.

% \emph{Is it a soft type system?} 

\emph{Significance:} the paper demonstrates that a first-order type system
designed to remove first-order casts can quickly remove a number of casts in
Scheme programs.


\end{document}
